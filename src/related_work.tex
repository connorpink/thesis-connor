\chapter{Related Work}
In a related work section, you discuss other work in your area that aims to solve your research problem or related research problems. The goal of this is not to discredit these papers but rather to show how your research builds on them, where you have used them for inspiration, how your approach differs, and how different approaches may even be combined to create an even more effective solution (remember the point about the discussion section above that every approach will have weaknesses, including yours). Keep it nice and friendly!
If you have read a few research papers, you might be wondering why we mention this section near the end. You will often find it in the front of the paper, e.g., after the background section. That can make sense for some research questions, but it will often be boring for a reader to sit through pages of related work before understanding what your idea actually is. Instead, put it at the end where the reader has a full understanding of your approach, its effects (through evaluation), and weaknesses (through your discussion).