\chapter{Idea}\label{ch:Idea}
In this chapter, you present your idea for answering your research question.
Keep in mind that your idea does not need to be ground-breaking and exceptionally novel. 
Instead, this thesis should serve as proof that you can independently gather an understanding of your research area, identify a sensible research questions and scientifically answer this question.

Your idea for solving your research question can come in many form depending on which research area you delve in.
It may be an algorithm, a type system, a study, a combination or something else entirely.
By talking to your advisor and getting to know your research topic by reading papers, you will most certainly gather an understanding of how to tackle your research questions.

Briefly describing your idea for answering the research questions can be difficult.
Be sure not to get lost in implementation details here, but rather just convey the overall idea. 
For example, it is of no interest here what programming language you use.
Also, do not mix the discussion and evaluation with describing your idea.
Try not to justify your idea here, but instead confidently describe it.
There will be plenty of space for discussion in later chapters. 
