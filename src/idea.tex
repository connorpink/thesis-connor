\chapter{Idea}\label{ch:Idea}
In this chapter, you introduce your idea: What solves your research question?
Your grade will most likely not be determined by how “great” your idea is, as you will have discussed it in detail before with your advisor – remember that this is not an engineering report. You will be graded mainly on your execution of the thesis. Do not feel pressured to have something novel or surprising here, it is often best to stick with what you and your advisor came up with together (which – due to that reason – will usually already be sufficiently novel).
The idea you present here can take many forms: it can be a study design (e.g., a benchmark that you want to perform to measure overheads), a system architecture (e.g., for a new database), or an algorithm (both an entirely new algorithm to solve a hitherto unsolved problem or the application of an exist- ing algorithm to a new problem).
Getting this section right can be difficult, especially as there can be some overlap with the next one.
Here is some advice regarding mistakes that students commonly make on their first try:

\paragraph{Do not include implementation details} When you write a systems paper, e.g., your idea is to write a new kind of database system, do not include your implementation of this system here. For example, for the design of your database system, it does not matter whether it was written in Java or C++ or which classes you have written. What matters is its overall design. Again, remember that your research question is not an engineering problem. One exception is if you require a certain technology for your approach, e.g., you investigate gcc and your approach does not extend to other compilers.

\paragraph{Do not include how you arrived at your solution} To find an approach for your research question, you will often start with a simple one and iterate on that to improve it. While this is a useful technique, it does not translate well to your manuscript. What you will end up with is not one approach but multiple, which makes it hard for your reader to understand quite what is going on. There can be some cases where you want to compare different approaches to the same research question, but that slightly changes your research question: Instead of a How should I solve it?, the question then is Which of the approaches A, B, and C is the best one?. Your approach section will then often describe how you plan to execute your comparison study, while the different approaches are in your background (if in doubt, ask your advisor). Focusing on a single approach and presenting that clearly makes it much easier for your readers to follow along and gives your thesis a clear message.

\paragraph{Do not mix evaluation and approach} You will evaluate your approach in your thesis (and we will get to that), but your evaluation should be in a different section. This includes the design of your evaluation, e.g., how you want to measure that your approach is good.
