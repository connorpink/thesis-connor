\chapter{Evaluation}\label{sec:evaluation}
After you have presented your approach, it is time to show why it is so great with an evaluation. An evaluation can take many forms, from a formal proof of an algorithm to a benchmark of an implementation. Finding the best evaluation method is something to discuss with your advisor because it depends on your research question and what you have done so far. Here are some ideas for what you can do:

\paragraph{Simulation} A simulation is a good way to evaluate your approach efficiently in a controlled environment. You can simply write a small simulation environment (which will often be faster than using any of the existing, full-fledged simulation frameworks) and plug in a dataset or algorithm. You will likely need some existing approach or other baseline to compare your new approach to. Be sure to determine some metrics that you want to measure, so you get relevant results.
\paragraph{Implementation} An implementation of a system can be helpful as well, al- though it is mostly coupled with some tests or benchmarks that show that the system does what it is supposed to do or improves an existing approach. Keep in mind that benchmarks are (usually) only meaningful when comparing to some- thing that already exists. Include an overview of how you implemented your system but do not go into too much detail when it is not necessary.
\paragraph{Formal Proof} You can also include a formal proof for your approach if that is the best way to show its correctness (very unlikely in our group).

In all cases, make sure to introduce your study design explicitly at the beginning. For simulation and benchmarking, be smart about what and how you measure: State your expectations and try to think like an adversary by coming up with scenarios that might break your approach. Also, when you notice that a result does not match your expectations, try to explain it and run additional experiments.