\chapter{Overview}\label{ch:overview}
The overview/background section normally serves two tasks: First, it introduces the terminology and definitions you use in the rest of your paper.
For example, there may be competing definitions on what a kernel is: the thing that runs your Linux or the thing that makes popcorn.
This is your opportunity to set the record straight, ideally using some cites.
Second, the background section lets you introduce the concepts you will use in the rest of your paper.
Every non-trivial piece of information should be mentioned here. It can be hard to decide what is considered trivial vs. non-trivial.
Think about who reads your paper (e.g., your advisor) and try not to give too much information as that would make your paper boring.
As a rule of thumb, everything that you learned in compulsory lectures can be considered common knowledge in your field, everything that will be new to most recent master’s graduates is non-trivial.
A background section in a thesis also serves another, indirect task: It shows the reader that you as a student have completely understood the research area you operate in.
Conversely, if, during your writing process, you or your advisor notice that there is something missing from this section, you should go back into the literature and try to close that gap.
